\documentclass[12pt]{article}

\usepackage[russian]{babel}
\usepackage[utf8]{inputenc}

\title{Домашняя работа № 1}
\author{Ланько Вадим}
\date{}

\begin{document}
	\maketitle
	\begin{flushright}\textit{Audi multa,\\
loquere pauca}\hspace{0pt}
\end{flushright}
\vspace{20pt}
\par{Это мой первый документ в системе компьютерной вёрстки \LaTeX.}
\begin{center}
\huge{\textsf{<<Ура!!!>>}}
\end{center}

А теперь формулы. \textsc{Формула} — краткое и точное словесное выражение, определение или же ряд математических величин, выраженный
условными знаками.
\vspace{15pt}\\
\hspace*{28pt}\textbf{\Large{Термодинамика}}

Уравнение Менделеева--Клапейрона~--- уравнение состояния идеального газа, имеющее вид $pV = \nu RT$, где $p$~--- давление, $V$~--- объем, занимаемый газом, $T$~--- температура газа, $\nu$~--- количество вещества газа, а
$R$~--- универсальная газовая постоянная.
\vspace{15pt}\\
\hspace*{28pt}\textbf{\Large{Геометрия} \hfill \Large{Планиметрия}}\hspace{0pt}

 Для \textsl{плоского} треугольника со сторонами $a, b, c$ и углом $\alpha$, лежащим
против стороны $a$, справедливо соотношение
\[a^2 = b^2 + c^2 - 2bc\cos \alpha,\]
из которого можно выразить косинус угла треугольника:
\[\cos a=\frac{b^2 + c^2 - a^2}{2bc}.\]
\newpage

Пусть $p$~--- полупериметр треугольника, тогда путем несложных преобразований можно получить, что
\[\tg \frac{\alpha}{2}=\sqrt{\frac{(p-b)(p-c)}{p(p-a)}},\]
\vspace*{1cm}\\
\hspace*{0pt}На сегодня, пожалуй, хватит$\ldots$ Удачи!





\end{document}